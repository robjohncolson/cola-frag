\documentclass{article}
\usepackage{geometry}
\usepackage{xcolor}
\usepackage{titlesec}
\usepackage{enumitem}
\usepackage{fancyhdr}
\usepackage{booktabs}
\usepackage{array}

% Defining colors
\definecolor{colaRed}{RGB}{201, 37, 44}     % Deep red for emphasis
\definecolor{colaBrown}{RGB}{58, 29, 20}    % Dark brown for text
\definecolor{colaCream}{RGB}{253, 246, 227} % Light cream background
\definecolor{colaDark}{RGB}{45, 12, 7}      % Deep dark brown
\definecolor{violetPurple}{RGB}{138, 93, 150} % Violet flower color

% Setting page geometry
\geometry{margin=1in}

% Configuring header and footer
\pagestyle{fancy}
\fancyhf{}
\renewcommand{\headrulewidth}{0.4pt}
\renewcommand{\footrulewidth}{0.4pt}
\fancyhead[C]{\textcolor{violetPurple}{\textbf{Peach Tea Fragrance}}}
\fancyfoot[C]{\textcolor{colaBrown}{\thepage}}

% Formatting section titles
\titleformat{\section}
  {\normalfont\Large\bfseries\color{colaRed}}
  {\thesection}{1em}{}

% Beginning document
\begin{document}

\begin{center}
\textcolor{violetPurple}{\LARGE\textbf{Peach Tea}}\\[0.5cm]
\textcolor{colaBrown}{\large\textit{A natural tea field with dry fruity-floral warmth, meadow citrus, and resinous chypre depth}}\\[0.5cm]
\end{center}

\section*{Naturals-Only 1ml Fragrance Recipe}

\begin{center}
\begin{tabular}{p{6.5cm}r}
\toprule
\textcolor{colaRed}{\textbf{Ingredient}} & \textcolor{colaRed}{\textbf{Amount}} \\
\midrule
\multicolumn{2}{l}{\textcolor{violetPurple}{\textbf{Order of Addition (Bottom to Top):}}} \\
\midrule
IPM (Carrier) & 740 microliters \\
3\% Deer Musk Tincture & 20 microliters \\
10\% Boronia Absolute (in IPM) & 15 microliters \\
10\% Black Currant Bud Absolute (in IPM) & 10 microliters \\
10\% Violet Leaf Absolute (in IPM) & 10 microliters \\
10\% Vanilla Absolute (in IPM) & 15 microliters \\
Frankincense Frereana EO & 12 microliters \\
Neroli EO & 18 microliters \\
Pink Pepper EO (undiluted) & 10 microliters \\
Coriander EO & 15 microliters \\
Lime EO & 20 microliters \\
Lemon EO & 20 microliters \\
Bergamot EO & 35 microliters \\
\bottomrule
\end{tabular}
\end{center}

\vspace{0.5cm}

\section*{Concentration Analysis}
\begin{itemize}[leftmargin=*]
  \item \textcolor{colaRed}{\textbf{Citrus Notes (7.5\%):}}
  \begin{itemize}
    \item Bergamot EO: 35 microliters (3.5\%)
    \item Lemon EO: 20 microliters (2\%)
    \item Lime EO: 20 microliters (2\%)
  \end{itemize}
  
  \item \textcolor{colaRed}{\textbf{Spice Elements (2.5\%):}}
  \begin{itemize}
    \item Coriander EO: 15 microliters (1.5\%)
    \item Pink Pepper EO: 10 microliters (1\%)
  \end{itemize}
  
  \item \textcolor{violetPurple}{\textbf{Floral-Tea-Fruity-Resin Notes (4.5\%):}}
  \begin{itemize}
    \item 10\% Boronia Absolute: 15 microliters (0.15\% final)
    \item 10\% Black Currant Bud Absolute: 10 microliters (0.1\% final)
    \item 10\% Violet Leaf Absolute: 10 microliters (0.1\% final)
    \item 10\% Vanilla Absolute: 15 microliters (0.15\% final)
    \item Neroli EO: 18 microliters (1.8\%)
    \item Frankincense Frereana EO: 12 microliters (1.2\%)
    \item 3\% Deer Musk Tincture: 20 microliters (0.06\% final)
  \end{itemize}
  
  \item \textcolor{colaBrown}{\textbf{Carrier:}}
  \begin{itemize}
    \item Isopropyl Myristate (IPM): 740 microliters (74.0\%)
  \end{itemize}
\end{itemize}

\section*{Expected Scent Profile}

\paragraph{\textcolor{colaRed}{\textbf{Top Notes (5-30 minutes):}}}
A crisp burst of bergamot, lemon, and lime, with pink pepper’s fruity-spicy zing and violet leaf’s subtle purple-green fizz, evokes a sunlit meadow with a dry, citrusy breeze, humble yet vibrant like a Kyushu morning.

\paragraph{\textcolor{colaBrown}{\textbf{Heart Notes (30 min - 3 hours):}}}
Boronia’s pu-erh tea-like floralcy blooms alongside neroli’s radiant, honeyed warmth, lifted by coriander’s fresh, citrusy-spicy spark. Black currant adds a dry, tangy fruitiness, mimicking osmanthus’s peachy glow, creating a simple tea field with sly Rush-like complexity.

\paragraph{\textcolor{colaDark}{\textbf{Base Notes (3-6+ hours):}}}
Frankincense’s dry, resinous warmth blends with deer musk’s soft muskiness and a whisper of vanilla’s dry sweetness, grounded by boronia’s lingering floral trace. The base is understated, like a Scottish moor, yet rich with a resinous chypre depth, evoking quiet elegance.

\paragraph{\textcolor{violetPurple}{\textbf{Overall Performance:}}}
Peach Tea wears for 4-6 hours with moderate projection, a natural, artisanal blend that’s humble and dry, with fruity-floral tea warmth, meadow citrus, and resinous chypre depth. It’s a teacher’s scent—bold yet intimate, reflecting a half-Japanese, half-Scottish storyteller with Kyushu tea fields and Scottish moors, inspired by Gucci Rush’s coriander-rich chypre elegance.

\section*{Preparation Instructions}
\begin{enumerate}
  \item Begin with IPM carrier in a 1ml vial.
  \item Add each ingredient in the specified order, using separate micropipette tips for each.
  \item Warm viscous absolutes (e.g., boronia, black currant, violet leaf, vanilla) to 40-50°C for easier handling.
  \item Seal vial and roll gently to blend without introducing air bubbles.
  \item Allow to rest 24-48 hours in a cool, dark place for notes to meld.
  \item Test on skin or beard, noting coriander’s spicy freshness and frankincense’s resinous depth; adjust coriander (12-18 microliters) or black currant (8-12 microliters) if needed.
\end{enumerate}

\section*{Notes for Future Batches}
\begin{itemize}
  \item Use DPG for 10\% dilutions (boronia, black currant, violet leaf, vanilla) if available for smoother blending.
  \item Store black currant absolute (~1.9g remaining) in the fridge; store frankincense, neroli, pink pepper, coriander, lime, lemon, bergamot in a cool, dark place.
  \item If coriander feels too green, reduce to 12 microliters; if heart needs more Rush-like spice, increase to 18 microliters.
  \item If black currant is too cola-like, reduce to 8 microliters; if fruity heart is weak, increase to 12 microliters.
  \item If vanilla feels too sweet, reduce to 10 microliters; if base needs warmth, increase to 20 microliters.
  \item Warm viscous absolutes to 40-50°C for easier handling.
\end{itemize}

\end{document}